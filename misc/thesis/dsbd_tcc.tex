%-----------------------------------------------------------------------
% Arquivo template *.tex para TCC da Especialização em Data Science &
% Big Data.
%-----------------------------------------------------------------------

% Classe do documento.
\documentclass[9pt, a4paper, twocolumn]{article}

%-----------------------------------------------------------------------
% Metadados do documento. ----------------------------------------------

\title{Título do trabalho de conclusão de curso de especialização}
\author{
  Gabriel Yuri Silva Ribeiro\footnotemark[1]\\
  Pedro Augusto de Lima e Silva\footnotemark[2]\\
  Luis C. E. Bona\footnotemark[3]\\
}

% Define as chamadas dos autores.
\def\articlefootnotes{
  \let\thefootnote\relax\footnotetext{$^1$Aluno do programa de Especialização em Data Science \& Big Data, \url{gyribeiro2014@gmail.com}.}
  \let\thefootnote\relax\footnotetext{$^2$Analista de dados da Gamersclub, \url{pdrals16@gmail.com}.}
  \let\thefootnote\relax\footnotetext{$^3$Professor do Departamento de Informática - DInf/UFPR.}
}
\date{}

% Define variáveis usadas no `dsbd_capa.tex`.
\def\capaautor{Gabriel Yuri Silva Ribeiro}
\def\capaorientador{Orientador do Programa}
\def\capaano{2022}
\makeatletter
\let\capatitulo\@title
\makeatother

%-----------------------------------------------------------------------
% Preâmbulo. -----------------------------------------------------------

% Define variáveis usadas no `dsbd_preamble.tex`.
\def\pathtologoufpr{src/logo-ufpr.png}
\def\pathtologodsbd{src/dsbd1x4.png}

% Não alterar o conteúdo do arquivo abaixo.
\input{src/dsbd_preamble.tex}

%-----------------------------------------------------------------------
% Início do documento. -------------------------------------------------

\begin{document}

%-----------------------------------------------------------------------
% Capa.

% Não alterar o conteúdo do arquivo abaixo.
\input{src/dsbd_capa.tex}

%-----------------------------------------------------------------------

\twocolumn

\maketitle

% Para criar a chamada dos autores no rodapé.
\articlefootnotes

\begin{abstract}
  Coloque aqui o resumo do trabalho de conclusão de curso. O resumo deve

  \lipsum[1]

  \noindent\textbf{Palavras-chave}: Análise X, método Y.
\end{abstract}

\renewcommand{\abstractname}{Abstract}
\begin{abstract}
  \it

  Versão em inglês do resumo. \lipsum[2]

  \noindent\textbf{Keywrods}: Versão em inglês, das
  palavras-chave.
\end{abstract}

%-----------------------------------------------------------------------

\section{Introdução}

\lipsum[3-5]

\begin{itemize}
\item Essa é uma lista de tópicos apenas para ilustrar como construir.
\item Coloquei apenas do itens que é o suficiente para você entender.
  \begin{itemize}
  \item Mas se a lista for hierárquica, então é só repetir o ambiente.
  \item Bem fácil.
  \end{itemize}
\end{itemize}

Fique atento aos seguintes símbolos

\begin{enumerate}
\item Para graus: as temperaturas usadas foram 25$^{\circ}$C e 40$^{\circ}$C.
\item Para números cardinais: as avaliações foram feitas no
  7\textsuperscript{\underline{o}} e 15\textsuperscript{\underline{o}}
  dias.
\end{enumerate}

Para referências bibliográficas, use \texttt{citet} para citação direta
no texto e \texttt{citep} para citação ao final de parágrafos. Confira os
exemplos.

Segundo \citet{wickham2016r}, blá blá blá. Tal coisa e coisa tal
\citep{bruce2019estatistica}. Por outro lado, \citet{grus2019data}
indica que blá blá blá. Estudos dessa natureza já foram relatados
\citep{bruce2019estatistica, fawcett2018data}.

%-----------------------------------------------------------------------

\section{Materiais e Métodos}

\lipsum[6]

\subsection{O conjunto de dados}

\lipsum[7]

\begin{table}[H]
  \caption{Dicionário do conjunto de dados.}
  \begin{tabular}{m{2.5cm} m{5cm}}
    %     \begin{tabular}{ll}
    \hline
    Variável & Descrição \\ \hline
    Renda & (contínua) Renda mensal do cliente, em reais. \\
    E.g. &  R\$ 10000, R\$ 4500.\\ \hline
    Dependentes & (discreta) Número de dependentes. \\
    E.g. & 0, 2, 5. \\ \hline
    Ecivil & (categórica) Estado civil do cliente. \\
    E.g. & casado, solteiro, divorciado, etc. \\ \hline
  \end{tabular}
\end{table}

\subsection{Limpeza e preparo dos dados}

\lipsum[8]

% \onecolumn
\begin{figure}[H]
  \centering
  \includegraphics[width = 7.5cm]{example-image}
  \caption{Legenda da figura. Exemplo: fluxograma de preparo dos dados
    considerando desde a extração do banco de dados, limpeza e imputação
    para emprego dos mesmos nos modelos.}
  \label{fig_01}
\end{figure}
% \twocolumn

\subsection{Modelos empregados}

\lipsum[9]

\subsubsection{Modelo A}

\lipsum[10]

\subsubsection{Modelo B}

\lipsum[11]

\subsection{Avaliação}

\lipsum[12]

\subsection{Métricas}

A \emph{acurácia} dá a proporção de predições corretas.
$$
\text{Acurácia: } A = \frac{TP + TN}{TP + FP + TN + FN}.
$$

A \emph{sensibilidade} mede a força do modelo prever um resultado
positivo.
$$
\text{Sensibilidade: } R = \frac{TP}{TP + FN}.
$$

A \emph{especificidade} mede a força do modelo prever um resultado
negativo.
$$
\text{Especificidade: } E = \frac{TN}{TN + FP}.
$$

A \emph{exatidão} mede a precisão de um resultado previsto como
positivo.
$$
\text{Exatidão: } P = \frac{TP}{TP + FP}.
$$

O \emph{F1} é a média harmônica entre precisão e a sensibilidade.
$$
\text{F1: } F = 2\cdot\frac{P\cdot R}{P + R}.
$$

%-----------------------------------------------------------------------

\section{Resultados e Discussões}

\lipsum[13]

\subsection{Ajuste dos modelos}

\lipsum[14]

\begin{table}[H]
  \caption{Hiperparâmetros ajustados e tempo de execução para o ajuste dos modelos aos dados.}
  \begin{tabular}{p{1.5cm} p{4.5cm} p{3em}}
    \hline
    Modelo & Hiperparâmetros & Tempo (min) \\ \hline
    SVR & 39 vetores de suporte. & 5 \\[1ex]
    Random forest & {$p = 10$ variáveis por árvore\newline
                    $m = 300$ árvores treinadas} & 15\\[4ex]
    Reg. Log. LASSO & {$\alpha = 0.001$} & 1\\ \hline
  \end{tabular}
\end{table}

\subsection{Medidas de performance}

\lipsum[15]

% \onecolumn
\begin{figure}[H]
  \centering
  \includegraphics[width = 7.5cm]{example-image}
  \caption{Legenda da figura. Exemplo: acurácia no conjunto de treino para os modelos treinados.}
  \label{fig_01}
\end{figure}
% \twocolumn

%-----------------------------------------------------------------------

\section{Conclusões}\label{Conclusao}

\lipsum[16]

\section*{Agradecimentos}

\lipsum[17]

%-----------------------------------------------------------------------
% Referências bibliográficas. ------------------------------------------

\bibliographystyle{unsrt}%
\bibliography{references.bib}%

\end{document}

%-----------------------------------------------------------------------